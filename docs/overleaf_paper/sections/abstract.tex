% ===== Abstract =====
\begin{center}
\parbox{0.92\textwidth}{
\textbf{Abstract} ---
Visually-rich Document Understanding은 문서 내 레이아웃 구조, 시각적 표현, 읽기 흐름을 종합적으로 이해해야 하는 고난도 문제이다.
본 연구에서는 2025 Samsung AI Challenge를 대상으로, Layout Detection, Reading Order Prediction, OCR을 통합 수행하는 On-device 문서 이해 시스템을 제안한다.
핵심 기여는 다음과 같다:
(1)~문서 방향에 따라 사전학습 모델과 파인튜닝 모델을 자동 선택하는 적응형 이중 모델 전략(Adaptive Dual-Model Strategy),
(2)~반복적 오탐지 패턴을 교정하는 규칙 기반 후처리,
(3)~인간의 독서 습관을 모방한 세 가지 구조 기반 읽기 순서 알고리즘(책 읽기, 신문 읽기, 포스터 읽기),
(4)~DPI 최적화와 경량 OCR을 통한 추론 효율성 확보.
제안 시스템은 Baseline 대비 Public Score \textbf{292\% 향상}(0.1168$\rightarrow$0.4577)을 달성하였으며,
OCR 추론 속도에서 \textbf{2.7$\times$} 개선을 기록하였다.
전체 모델 크기는 221\,MB로 On-device 배포가 가능하다.
\\[0.6em]
\textbf{Keywords:}
Document Understanding, Layout Detection, Reading Order, OCR, YOLOv12, On-device AI
}
\end{center}
