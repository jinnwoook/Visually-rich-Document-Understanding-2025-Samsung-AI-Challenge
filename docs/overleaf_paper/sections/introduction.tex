% ===== 1. Introduction =====
\section{Introduction}
\label{sec:intro}

현대 문서는 텍스트뿐만 아니라 제목, 표, 수식, 이미지, 강조 표현 등 다양한 시각적 요소로 구성된다.
인간은 이러한 요소를 종합적으로 해석하여 문서의 의미와 의도를 이해하지만,
기존 OCR 중심의 문서 처리 기술은 텍스트 추출에 머무르며
문서 구조와 강조 의도를 충분히 반영하지 못하는 한계를 가진다~\cite{pfitzmann2022doclaynet}.

Samsung AI Challenge의 Visually-rich Document Understanding 트랙은
이러한 한계를 극복하기 위해 문서 내 \textbf{객체 탐지(Layout Detection)},
\textbf{텍스트 추출(OCR)}, \textbf{읽기 순서(Reading Order)}를 통합적으로 평가한다.
세 모듈의 가중 합산 점수(Layout Detection 35\%, Reading Order 35\%, OCR 30\%)로 최종 성능을 산출하며,
각 모듈은 순차 파이프라인으로 연결되므로 상위 단계의 오류가 하위 단계 성능에 직접 전파된다.

본 연구의 목표는 \textbf{정확도와 추론 속도를 동시에 고려한
On-device 환경용 문서 이해 시스템}을 구축하는 것이다.
이를 위해 문서 형태에 따른 적응형 모델 선택, 규칙 기반 후처리,
구조 중심의 읽기 순서 알고리즘, 경량 OCR 최적화를 결합한 종합 파이프라인을 설계하였다.

본 논문의 주요 기여는 다음과 같다:
\begin{enumerate}[label=(\arabic*)]
    \item \textbf{적응형 이중 모델 전략}:
    세로형 문서(PDF, 논문)와 가로형 문서(PPTX, 포스터)의 구조적 차이를 인식하고,
    각 유형에 최적화된 모델을 자동 선택한다.
    \item \textbf{구조 기반 읽기 순서 알고리즘}:
    의미론적 분석 없이 레이아웃 구조만으로 인간의 독서 패턴을 모방하는
    세 가지 읽기 전략(책$\cdot$신문$\cdot$포스터 읽기)을 설계한다.
    \item \textbf{Optuna 기반 신뢰도 임계값 최적화}:
    클래스별 confidence threshold를 대회 점수를 목적 함수로 자동 탐색하여
    탐지 성능을 극대화한다.
    \item \textbf{On-device 경량 시스템}:
    221\,MB 모델 크기와 EasyOCR 기반 경량 OCR로 실용적 배포가 가능한 시스템을 구현한다.
\end{enumerate}
