% ===== 2. Related Work =====
\section{Related Work}
\label{sec:related}

\subsection{Document Layout Analysis}

문서 레이아웃 분석은 문서 이미지에서 텍스트, 표, 그림 등의 영역을 탐지하고 분류하는 과제이다.
초기 연구는 규칙 기반 또는 전통적 컴퓨터 비전 기법에 의존하였으나,
딥러닝의 발전과 함께 객체 탐지 프레임워크 기반 접근법이 주류가 되었다.

\textbf{DocLayNet}~\cite{pfitzmann2022doclaynet}은 IBM에서 공개한 대규모 문서 레이아웃 데이터셋으로,
80,000장 이상의 문서에 대해 11개 카테고리의 어노테이션을 제공한다.
\textbf{YOLOv12}~\cite{ultralytics}는 Ultralytics에서 개발한 실시간 객체 탐지 프레임워크로,
DocLayNet으로 사전학습된 YOLOv12-Large 모델은
문서 도메인에서 높은 탐지 성능과 빠른 추론 속도를 동시에 제공한다.

\subsection{Optical Character Recognition}

OCR은 문서 이미지에서 텍스트를 추출하는 기술로,
텍스트 탐지(detection)와 텍스트 인식(recognition)의 두 단계로 구성된다.
\textbf{CRAFT}~\cite{baek2019craft}는 문자 수준의 영역 인식을 통해
텍스트 영역을 탐지하는 모델이며,
\textbf{EasyOCR}~\cite{easyocr}은 80개 이상의 언어를 지원하는
경량 OCR 엔진으로 CRAFT를 텍스트 탐지 백엔드로 활용한다.

\subsection{Reading Order Prediction}

문서 내 요소의 읽기 순서를 결정하는 문제는
단순 좌표 정렬로는 해결할 수 없는 복잡한 과제이다.
특히 멀티 컬럼 레이아웃, 포스터형 자유 배치, 표$\cdot$그림$\cdot$캡션의 상호작용 등을 고려해야 한다.
기존 연구는 주로 그래프 기반 방법이나 Transformer 기반 순서 예측을 시도하였으나,
본 연구에서는 추가 학습 없이 레이아웃 구조만으로 작동하는
규칙 기반 읽기 순서 알고리즘을 제안한다.

\subsection{Hyperparameter Optimization}

\textbf{Optuna}~\cite{akiba2019optuna}는 효율적인 하이퍼파라미터 최적화 프레임워크로,
Tree-structured Parzen Estimator (TPE) 등의 알고리즘을 통해
목적 함수를 최적화한다.
본 연구에서는 Optuna를 활용하여 클래스별 detection confidence threshold를
대회 Public Score를 목적 함수로 자동 탐색하였다.
